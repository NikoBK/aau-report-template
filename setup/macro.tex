% Define built in chapters to respect danish/english setting
\if\useenglish0
  \newcommand{\prefacename}{Forord}
  \newcommand{\introductionname}{Indledning}
  \newcommand{\analysisname}{Problemanalyse}
  \newcommand{\definitionname}{Problemafgrænsning}
  \newcommand{\formulationname}{Problemformulering}
  \newcommand{\solutionname}{Løsning}
  \newcommand{\listingname}{Kodeuddrag}
\else
  \newcommand{\prefacename}{Preface}
  \newcommand{\introductionname}{Introduction}
  \newcommand{\analysisname}{Problem analysis}
  \newcommand{\definitionname}{Definition of the problem}
  \newcommand{\formulationname}{Formulation of the problem}
  \newcommand{\solutionname}{Solution}
  \newcommand{\listingname}{Code snippet}
\fi

% newcommand/renewcommand convenience function
\newcommand{\neworrenewcommand}[1]{\providecommand{#1}{}\renewcommand{#1}}

% Set up lists
\setsepchar{,}
\readlist\authors{\authors}
\readlist\emails{\emails}
\readlist\supervisors{\supervisors}



% No indentation environment
\newlength\mystoreparindent
\newenvironment{noindentenv}{%
  \setlength{\mystoreparindent}{\the\parindent}%
  \setlength{\parindent}{0pt}%
}{%
  \setlength{\parindent}{\mystoreparindent}%
}

% Titlepage entry macros
\newcommand{\tpentry}[2]{\textbf{#1:}\\#2\bigskip\par}
\newcommand{\tpentryl}[2]{\textbf{#1:}\\#2}

% The danish titlepage
\newcommand{\titlepageda}{%
  \titlepaget%
  {AAU_STUDENTERRAPPORT_blue_rgb}%
  {\edunameda}%
  {Aalborg Universitet}%
  {Titel}{\titleda}%
  {Emne}{\subjectda}%
  {Projektperiode}{\projectperioden}%
  {Projektgruppe}%
  {Deltager(e)}%
  {Vejleder(e)}%
  {Oplagstal}%
  {Sidetal}%
  {Afleveringsdato}%
  {Synopse}%
  {abstractda}%
}

% The english titlepage
\newcommand{\titlepageen}{%
  \titlepaget%
  {AAU_UK_STUDENTREPORT_blue_rgb}%
  {\edunameen}%
  {Aalborg University}%
  {Title}{\titleen}%
  {Emne}{\subjecten}%
  {Project period}{\projectperioden}%
  {Project group}%
  {Participant(s)}%
  {Supervisor(s)}%
  {Copies}%
  {Pages}%
  {Due date}%
  {Abstract}%
  {abstracten}%
}

% Titlepage template
\newcommand{\titlepaget}[9]{%
\neworrenewcommand{\titlepagett}[8]{%
  \begin{multicols}{2}%
    \includegraphics{figures/#1}% AAU logo
    \columnbreak\\%
    {\raggedleft%
      \textbf{#2}\\% Education name
      #3\\% AAU title
      \href{https://www.aau.dk}{https://www.aau.dk}\\%
    }%
  \end{multicols}%
  \begin{multicols}{2}%
    \begin{noindentenv}%
      \tpentry{#4}{#5}% Title
      \tpentry{#6}{#7}% Semester theme
      \tpentry{#8}{#9}% Project period
      \tpentry{##1}{\projectgroup}% Project group
      \tpentry{##2}{\authorlist}% Participants
      \tpentry{##3}{\supervisorlist}% Supervisors
      \tpentry{##4}{\numcopies}% Copies
      \tpentry{##5}{\pageref{lastpage}}% Pages
      \tpentryl{##6}{\duedate}% Due date
    \end{noindentenv}%
    \columnbreak\\%
    \textbf{##7:}% Abstract
    \begin{mdframed}%
      \input{sections/##8}% Abstract text
    \end{mdframed}%
  \end{multicols}%
}%
\titlepagett%
}

% Clickable email
\newcommand{\emaillink}[1]{\href{mailto:#1}{#1}}

% Preface signature macro
\newcommand{\signature}[2]{%
  \rule{\linewidth}{0.5pt}\\%
  #1\\%
  {\footnotesize <#2>}\\%
}

% Generate a list of authors
\newcommand{\authorlist}{%
  \foreachitem\authoritem\in\authors{%
    \IfStrEq{\authoritem}{\authors[-1]}{\authoritem}{\authoritem\\}%
  }%
}

% Generate a comma seperated list of authors
% For use in hyperref setup
\def\authorcomma{}
\newcommand\authoradd[1]{\edef\authorcomma{\authorcomma{}#1}}
\foreachitem\authoritem\in\authors{%
  \IfStrEq{\authoritem}{\authors[-1]}{%
    \authoradd{\authoritem}%
  }{%
    \authoradd{\authoritem, }%
  }%
}


\author{\projectgroup}

% Generate a list of supervisors
\newcommand{\supervisorlist}{%
  \foreachitem\supervisoritem\in\supervisors{%
    \IfStrEq{\supervisoritem}{\supervisors[-1]}{\supervisoritem}{\supervisoritem\\}%
  }%
}

% Generate the list of signatures on preface.tex
\newcommand{\signaturelist}{%
  \vspace{3\baselineskip}%
  \foreachitem\signatureitem\in\authors{%
    \ifthenelse{\isodd{\signatureitemcnt{}}}%
      {%
        \edef\oldsignatureitem{\signatureitem}%
        \edef\oldsignatureitemcnt{\signatureitemcnt}%
        \IfStrEq{\signatureitem}{\authors[-1]}{%
          \begin{center}%
            \signature{\signatureitem}{%
              \foreachitem\emailitem\in\emails{%
                \IfStrEq{\emailitemcnt{}}{\signatureitemcnt{}}{\emaillink\emailitem}{}%
              }%
            }%
          \end{center}%
        }{}%
      }%
      {%
        \begin{multicols}{2}%
          \centering%
           \signature{\oldsignatureitem}{%
            \foreachitem\emailitem\in\emails{%
              \IfStrEq{\emailitemcnt{}}{\oldsignatureitemcnt{}}{\emaillink\emailitem}{}%
            }%
          }%
          \columnbreak%
          \signature{\signatureitem}{%
            \foreachitem\emailitem\in\emails{%
              \IfStrEq{\emailitemcnt{}}{\signatureitemcnt{}}{\emaillink\emailitem}{}%
            }%
          }%
        \end{multicols}%
        \IfStrEq{\signatureitem}{\authors[-1]}{}{\vspace{1\baselineskip}}%
      }%
  }%
}

\if\useenglish0
  \newcommand{\langid}{da-DK}
\else
  \newcommand{\langid}{en-US}
\fi

\if\final0
  \newcommand{\critical}[1]{\todo[color=red]{#1}}
  \newcommand{\changeme}[1]{\todo[color=red!40]{#1}}
  \newcommand{\unsure}[1]{\todo[color=orange!40]{#1}}
  \newcommand{\info}[1]{\todo[color=blue!40]{#1}}
  \newcommand{\nitpick}[1]{\todo[color=gray!40, size=\small{}]{#1}}
  \newcommand{\missingref}[1]{\todo[color=green!40]{#1}}
\else
  \newcommand{\critical}[1]{\PackageError{aau-report-template}{You have todos in final mode}{Delete/fix your todos}}
  \newcommand{\changeme}[1]{\PackageError{aau-report-template}{You have todos in final mode}{Delete/fix your todos}}
  \newcommand{\unsure}[1]{\PackageError{aau-report-template}{You have todos in final mode}{Delete/fix your todos}}
  \newcommand{\info}[1]{\PackageError{aau-report-template}{You have todos in final mode}{Delete/fix your todos}}
  \newcommand{\nitpick}[1]{\PackageError{aau-report-template}{You have todos in final mode}{Delete/fix your todos}}
  \newcommand{\missingref}[1]{\PackageError{aau-report-template}{You have todos in final mode}{Delete/fix your todos}}
  \newcommand{\todo}[1]{\PackageError{aau-report-template}{You have todos in final mode}{Delete/fix your todos}}
  \newcommand{\missingfigure}[1]{\PackageError{aau-report-template}{You have a missing figure}{Delete/fix your figure}}
\fi
